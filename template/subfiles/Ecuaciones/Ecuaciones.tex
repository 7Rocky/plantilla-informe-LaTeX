\documentclass[../../main.tex]{subfile}

\begin{document}

  Usar ecuaciones en LaTeX es bastante sencillo. Si se quiere poner una ecuación en línea con el texto, se pone \mintinline{tex}|$\mathrm{e}^{\mathrm{i}\pi} + 1 = 0$|, y quedaría como $\mathrm{e}^{\mathrm{i}\pi} + 1 = 0$.

  Si se prefiere poner la ecuación en un nuevo párrafo y centrada, se debe poner la misma expresión pero con dos símbolos \$ de apertura y cierre. Es decir: 
  
  \mintinline{tex}|$$\sum_{n = 1}^\infty \frac{1}{n^2} = \frac{\pi^2}{6}$$|
  
  El resultado es:

  $$
  \sum_{n = 1}^\infty \frac{1}{n^2} = \frac{\pi^2}{6}
  $$

  Si se quiere referenciar una ecuación es necesario utilizar el entorno \texttt{equation}. Por ejemplo, el siguiente código:

  \begin{minted}[fontsize=\footnotesize]{tex}
\begin{equation}
  \label{eq:matrix_fibonacci}
  \begin{pmatrix}
    F_{n + 1} & F_n \\
    F_n & F_{n - 1}
  \end{pmatrix} = \begin{pmatrix}
    1 & 1 \\
    1 & 0
  \end{pmatrix} ^ n
\end{equation}
  \end{minted}

  Genera la siguiente ecuación:

  \begin{equation}
    \label{eq:matrix_fibonacci}
    \begin{pmatrix}
      F_{n + 1} & F_n \\
      F_n & F_{n - 1}
    \end{pmatrix} = \begin{pmatrix}
      1 & 1 \\
      1 & 0
    \end{pmatrix} ^ n
  \end{equation}

  Esta ecuación puede ser referenciada utilizando \mintinline{tex}|\eqref{eq:matrix_fibonacci}|, y queda como \eqref{eq:matrix_fibonacci}.

\end{document}
