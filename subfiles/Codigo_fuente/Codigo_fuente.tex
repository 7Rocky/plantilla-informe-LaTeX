\documentclass[../../main.tex]{subfile}

\providecommand{\subdirname}{subfiles/Codigo_fuente}

\begin{document}

  Para incluir código fuente se utiliza el paquete \texttt{minted}, el cual tiene soporte para gran cantidad de lenguajes de programación y lenguajes de marcado. El código puede ser incluido directamente en el código LaTeX, de manera que:

  \begin{minted}[escapeinside=||, fontsize=\footnotesize]{tex}
\begin{minted}[bgcolor=lightgray]{html}
<!doctype html>
<html>
  <head>
    <title>Hello, world!</title>
    <meta charset="utf-8">
  </head>
  <body>
    <h1>Hello, world!</h1>
  </body>
</html>
\end{|minted|}
  \end{minted}

  Genera el siguiente código coloreado sintácticamente:

  \begin{minted}[bgcolor=lightgray, fontsize=\footnotesize]{html}
<!doctype html>
<html>
  <head>
    <title>Hello, world!</title>
    <meta charset="utf-8">
  </head>
  <body>
    <h1>Hello, world!</h1>
  </body>
</html>
  \end{minted}

  El parámetro \texttt{bgcolor=lightgray} indica que el color de fondo sea gris claro. Este color \texttt{lightgray} está definido en el archivo \texttt{packages/mycommands.sty}.

  Al entorno \texttt{minted} se le pueden añadir muchos parámetros, entre ellos \texttt{linenos} (para que se indique el número de cada linea del código a la izquierda) y \texttt{fontsize} (para indicar el tamaño de letra, este tamaño suele no especificarse o ponerse como \mintinline{tex}|fontsize=\footnotesize| o como \mintinline{tex}|fontsize=\scriptsize|).

  Una utilidad muy buena de \texttt{minted} es la posibilidad de insertar código fuente desde un archivo existente, sin necesidad de copiar el código en el archivo de LaTeX. Además, se puede indicar desde qué línea hasta qué linea se debe incluir. Para ello, en esta plantilla existe el comando \mintinline{tex}|codecaption|, el cual requiere de cuatro parámetros:

  \begin{enumerate}
    \item Ruta al fichero.
    \item El lenguaje de programación o marcado utilizado.
    \item Texto del pie de código.
  \end{enumerate}

  A este comando se le pueden insertar muchas opciones, como \texttt{bgcolor}, \texttt{linenos}, \texttt{fontsize}, \texttt{firstline} y \texttt{lastline}. Por ejemplo, con la siguiente instrucción:

  \begin{minted}[fontsize=\footnotesize]{tex}
\codecaption[
  bgcolor=lightgray,
  firstline=28,
  fontsize=\footnotesize,
  lastline=36
]
{Lab.m}
{\matlab}
{Función para generar la sucesión de Fibonacci.}
  \end{minted}

  Se obtiene:

  \codecaption[bgcolor=lightgray, firstline=28, fontsize=\footnotesize, lastline=36]{Lab.m}{\matlab}{Función para generar la sucesión de Fibonacci.}

  Al igual que antes, para referenciar este código, se puede utilizar el comando configurado como \mintinline{tex}|\coderef{Lab.m}|, indicando la ruta del archivo, de manera que se obtiene el hipervínculo: \coderef{Lab.m}.

  De nuevo, en esta plantilla, el código fuente se debe poner en la carpeta \texttt{src/}. Si se quiere elegir otra carpeta independiente de la plantilla se ha de modificar el archivo \texttt{packages/codeconfig.sty}, donde dice:

  \begin{minted}[fontsize=\footnotesize]{tex}
% Ruta relativa al código fuente
\newcommand*{\sourcedir}{./src/}
  \end{minted}

  Es probable que sea necesario poner una ruta absoluta y evitar nombres de carpetas y archivos con espacios o caracteres raros.

  Si no se quiere referenciar el código o escribir un pie de código, se puede utilizar el comando \mintinline{tex}|\codenocaption|, el cual requiere los mismos argumentos que el comando anterior, a excepción del texto del pie de código. Por ejemplo:

  \begin{minted}[fontsize=\footnotesize]{tex}
\codenocaption[bgcolor=lightgray, fontsize=\footnotesize]{main.c}{c}
  \end{minted}

  Muestra el siguiente código en C:

  \codenocaption[bgcolor=lightgray, fontsize=\footnotesize]{main.c}{c}

  A la hora de elegir el lenguaje para que \texttt{minted} coloree el código, cabe mencionar que no todos funcionan bien al 100\%, puede que los \textit{lexers} no estén actualizados a las últimas tendencias de los lenguajes. Por este motivo, se han añadido algunos \textit{lexers} personalizados en la carpeta \texttt{lexers/}. Están escritos en Python y hacen uso de expresiones regulares. Para utilizarlos, basta con poner \mintinline{tex}|\dockerfile| en lugar de \texttt{dockerfile}, \mintinline{tex}|\json| en lugar de \texttt{json}, \mintinline{tex}|\yaml| en lugar de \texttt{yaml} y \mintinline{tex}|\matlab| en lugar de \texttt{matlab} (este último comando simplemente cambia el \textit{lexer} de \texttt{matlab} por el de \texttt{octave}, porque funciona mejor).

\end{document}
