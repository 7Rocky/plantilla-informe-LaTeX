\documentclass[../../main.tex]{subfile}

\providecommand{\subdirname}{subfiles/Estructura_del_documento}

\begin{document}

  Una bondad de utilizar LaTeX como generador de documentos frente a otras herramientas es la posibilidad de estructurar el código del documento completo en distintos archivos para poder trabajar mejor. Esto se realiza mediante un paquete llamado \texttt{subfiles}.

  La manera más común de utilizar \texttt{subfiles} es utilizar la siguiente línea en el documento que incluye el \texttt{subfile}:

  \begin{minted}[fontsize=\footnotesize]{tex}
\subfile{subfiles/Estructura_del_documento/Prueba_subfiles.tex}
  \end{minted}

  Y el resultado de este comando es el siguiente:

  \subfile{subfiles/Estructura_del_documento/Prueba_subfiles.tex}

  Este archivo \texttt{Prueba\_subfiles.tex} contiene lo siguiente:

  \codenocaption[bgcolor=lightgray, fontsize=\footnotesize]{../subfiles/Estructura_del_documento/Prueba_subfiles.tex}{tex}

  En cada \texttt{subfile} hay que incluir la ruta relativa desde el \texttt{subfile} hasta el archivo principal (\texttt{main.tex} en esta plantilla) y decir que el documento es de clase \texttt{subfile}.

  A continuación, se incluye una variable que guarda la ruta del \texttt{subfile} desde el archivo principal, para facilitar la inclusión de \texttt{subfiles} en el propio \texttt{subfile}. El comando descrito anteriormente podría haberse utilizado como:

  \begin{minted}[fontsize=\footnotesize]{tex}
\subfile{\subdirname/Prueba_subfiles.tex}
  \end{minted}

  Es importante notar que los hijos del archivo \texttt{main.tex} tiene una variable \mintinline{latex}|\dirname|. Los hijos de estos tienen una variable \mintinline{latex}|\subdirname|. Y, a su vez, los hijos de estos tienen una variable \mintinline{latex}|\subsubdirname|. Nótese cómo con cada nuevo nivel de \texttt{subfiles} se añade un \texttt{sub} al nombre de la variable.

  Como se ha dicho antes, esto es para facilitar el uso de \texttt{subfiles} y no tener que escribir siempre la ruta relativa completa desde el archivo \texttt{main.tex}. Se hace más estructurado el proyecto LaTeX al trabajar con carpetas y subcarpetas, todas organizadas según las distintas secciones del documento.

  Así, el uso que se recomienda de los \texttt{subfiles} es el de un \texttt{subfile} por apartado en el documento\footnote{También es conveniente utilizar un \texttt{subfile} cuando se incluya un dibujo con \texttt{tikzpicture}, una tabla, una gráfica con \texttt{pgfplots}...}. De manera que en el archivo padre se tenga algo como lo siguiente:

  \begin{minted}[fontsize=\footnotesize]{tex}
\section{Estructura del documento}
  \label{sec:\dirname/Estructura_del_documento/Estructura_del_documento.tex}
  \subfile{\dirname/Estructura_del_documento/Estructura_del_documento.tex}
  \end{minted}

  Sin embargo, estas tres líneas de código se han resumido en un comando \mintinline{latex}|\mysection|, que recibe tres parámetros. El equivalente al código anterior sería:

  \begin{minted}[fontsize=\footnotesize]{tex}
\mysection{Estructura del documento}{Estructura_del_documento}{\dirname}
  \end{minted}

  Como se puede observar, el primer parámetro es el título que se mostrará en el documento; el segundo parámetro es el nombre del archivo que tiene el contenido de la sección (sin extensión); y el tercer argumento es la variable que se provee al inicio de cada \texttt{subfile}, de la que se ha hablado anteriormente.

  Al igual que existe \mintinline{latex}|\mysection| en el archivo \texttt{packages/titlesconfig.sty}, existen otros comandos como \mintinline{latex}|\mysubsection| y \mintinline{latex}|\mysubsubsection|, los cuales tienen la misma utilidad pero con otros niveles de sección en el documento.

  El ejemplo más claro de cómo utilizar \texttt{subfiles} y de cómo estructurar el código LaTeX es el presente documento y su código. Analícese detenidamente para mayor comprensión.

\end{document}
